% Options for packages loaded elsewhere
\PassOptionsToPackage{unicode}{hyperref}
\PassOptionsToPackage{hyphens}{url}
%
\documentclass[
  man]{apa6}
\usepackage{amsmath,amssymb}
\usepackage{lmodern}
\usepackage{iftex}
\ifPDFTeX
  \usepackage[T1]{fontenc}
  \usepackage[utf8]{inputenc}
  \usepackage{textcomp} % provide euro and other symbols
\else % if luatex or xetex
  \usepackage{unicode-math}
  \defaultfontfeatures{Scale=MatchLowercase}
  \defaultfontfeatures[\rmfamily]{Ligatures=TeX,Scale=1}
\fi
% Use upquote if available, for straight quotes in verbatim environments
\IfFileExists{upquote.sty}{\usepackage{upquote}}{}
\IfFileExists{microtype.sty}{% use microtype if available
  \usepackage[]{microtype}
  \UseMicrotypeSet[protrusion]{basicmath} % disable protrusion for tt fonts
}{}
\makeatletter
\@ifundefined{KOMAClassName}{% if non-KOMA class
  \IfFileExists{parskip.sty}{%
    \usepackage{parskip}
  }{% else
    \setlength{\parindent}{0pt}
    \setlength{\parskip}{6pt plus 2pt minus 1pt}}
}{% if KOMA class
  \KOMAoptions{parskip=half}}
\makeatother
\usepackage{xcolor}
\IfFileExists{xurl.sty}{\usepackage{xurl}}{} % add URL line breaks if available
\IfFileExists{bookmark.sty}{\usepackage{bookmark}}{\usepackage{hyperref}}
\hypersetup{
  pdftitle={On the interplay of motivational characteristics and school grades: The role of Need for Cognition},
  pdfauthor={Anja Strobel1*, Alexander Strobel2*, Franzis Preckel3, \& Ricarda Steinmayer4},
  pdflang={en-EN},
  pdfkeywords={Need for Cognition, Grades, Academic Self-Concept, Latent Change Score Modeling, Longitudinal},
  hidelinks,
  pdfcreator={LaTeX via pandoc}}
\urlstyle{same} % disable monospaced font for URLs
\usepackage{graphicx}
\makeatletter
\def\maxwidth{\ifdim\Gin@nat@width>\linewidth\linewidth\else\Gin@nat@width\fi}
\def\maxheight{\ifdim\Gin@nat@height>\textheight\textheight\else\Gin@nat@height\fi}
\makeatother
% Scale images if necessary, so that they will not overflow the page
% margins by default, and it is still possible to overwrite the defaults
% using explicit options in \includegraphics[width, height, ...]{}
\setkeys{Gin}{width=\maxwidth,height=\maxheight,keepaspectratio}
% Set default figure placement to htbp
\makeatletter
\def\fps@figure{htbp}
\makeatother
\setlength{\emergencystretch}{3em} % prevent overfull lines
\providecommand{\tightlist}{%
  \setlength{\itemsep}{0pt}\setlength{\parskip}{0pt}}
\setcounter{secnumdepth}{-\maxdimen} % remove section numbering
% Make \paragraph and \subparagraph free-standing
\ifx\paragraph\undefined\else
  \let\oldparagraph\paragraph
  \renewcommand{\paragraph}[1]{\oldparagraph{#1}\mbox{}}
\fi
\ifx\subparagraph\undefined\else
  \let\oldsubparagraph\subparagraph
  \renewcommand{\subparagraph}[1]{\oldsubparagraph{#1}\mbox{}}
\fi
\newlength{\cslhangindent}
\setlength{\cslhangindent}{1.5em}
\newlength{\csllabelwidth}
\setlength{\csllabelwidth}{3em}
\newlength{\cslentryspacingunit} % times entry-spacing
\setlength{\cslentryspacingunit}{\parskip}
\newenvironment{CSLReferences}[2] % #1 hanging-ident, #2 entry spacing
 {% don't indent paragraphs
  \setlength{\parindent}{0pt}
  % turn on hanging indent if param 1 is 1
  \ifodd #1
  \let\oldpar\par
  \def\par{\hangindent=\cslhangindent\oldpar}
  \fi
  % set entry spacing
  \setlength{\parskip}{#2\cslentryspacingunit}
 }%
 {}
\usepackage{calc}
\newcommand{\CSLBlock}[1]{#1\hfill\break}
\newcommand{\CSLLeftMargin}[1]{\parbox[t]{\csllabelwidth}{#1}}
\newcommand{\CSLRightInline}[1]{\parbox[t]{\linewidth - \csllabelwidth}{#1}\break}
\newcommand{\CSLIndent}[1]{\hspace{\cslhangindent}#1}
\ifLuaTeX
\usepackage[bidi=basic]{babel}
\else
\usepackage[bidi=default]{babel}
\fi
\babelprovide[main,import]{english}
% get rid of language-specific shorthands (see #6817):
\let\LanguageShortHands\languageshorthands
\def\languageshorthands#1{}
% Manuscript styling
\usepackage{upgreek}
\captionsetup{font=singlespacing,justification=justified}

% Table formatting
\usepackage{longtable}
\usepackage{lscape}
% \usepackage[counterclockwise]{rotating}   % Landscape page setup for large tables
\usepackage{multirow}		% Table styling
\usepackage{tabularx}		% Control Column width
\usepackage[flushleft]{threeparttable}	% Allows for three part tables with a specified notes section
\usepackage{threeparttablex}            % Lets threeparttable work with longtable

% Create new environments so endfloat can handle them
% \newenvironment{ltable}
%   {\begin{landscape}\begin{center}\begin{threeparttable}}
%   {\end{threeparttable}\end{center}\end{landscape}}
\newenvironment{lltable}{\begin{landscape}\begin{center}\begin{ThreePartTable}}{\end{ThreePartTable}\end{center}\end{landscape}}

% Enables adjusting longtable caption width to table width
% Solution found at http://golatex.de/longtable-mit-caption-so-breit-wie-die-tabelle-t15767.html
\makeatletter
\newcommand\LastLTentrywidth{1em}
\newlength\longtablewidth
\setlength{\longtablewidth}{1in}
\newcommand{\getlongtablewidth}{\begingroup \ifcsname LT@\roman{LT@tables}\endcsname \global\longtablewidth=0pt \renewcommand{\LT@entry}[2]{\global\advance\longtablewidth by ##2\relax\gdef\LastLTentrywidth{##2}}\@nameuse{LT@\roman{LT@tables}} \fi \endgroup}

% \setlength{\parindent}{0.5in}
% \setlength{\parskip}{0pt plus 0pt minus 0pt}

% Overwrite redefinition of paragraph and subparagraph by the default LaTeX template
% See https://github.com/crsh/papaja/issues/292
\makeatletter
\renewcommand{\paragraph}{\@startsection{paragraph}{4}{\parindent}%
  {0\baselineskip \@plus 0.2ex \@minus 0.2ex}%
  {-1em}%
  {\normalfont\normalsize\bfseries\itshape\typesectitle}}

\renewcommand{\subparagraph}[1]{\@startsection{subparagraph}{5}{1em}%
  {0\baselineskip \@plus 0.2ex \@minus 0.2ex}%
  {-\z@\relax}%
  {\normalfont\normalsize\itshape\hspace{\parindent}{#1}\textit{\addperi}}{\relax}}
\makeatother

% \usepackage{etoolbox}
\makeatletter
\patchcmd{\HyOrg@maketitle}
  {\section{\normalfont\normalsize\abstractname}}
  {\section*{\normalfont\normalsize\abstractname}}
  {}{\typeout{Failed to patch abstract.}}
\patchcmd{\HyOrg@maketitle}
  {\section{\protect\normalfont{\@title}}}
  {\section*{\protect\normalfont{\@title}}}
  {}{\typeout{Failed to patch title.}}
\makeatother

\usepackage{xpatch}
\makeatletter
\xapptocmd\appendix
  {\xapptocmd\section
    {\addcontentsline{toc}{section}{\appendixname\ifoneappendix\else~\theappendix\fi\\: #1}}
    {}{\InnerPatchFailed}%
  }
{}{\PatchFailed}
\keywords{Need for Cognition, Grades, Academic Self-Concept, Latent Change Score Modeling, Longitudinal\newline\indent Word count: }
\DeclareDelayedFloatFlavor{ThreePartTable}{table}
\DeclareDelayedFloatFlavor{lltable}{table}
\DeclareDelayedFloatFlavor*{longtable}{table}
\makeatletter
\renewcommand{\efloat@iwrite}[1]{\immediate\expandafter\protected@write\csname efloat@post#1\endcsname{}}
\makeatother
\usepackage{lineno}

\linenumbers
\usepackage{csquotes}
\ifLuaTeX
  \usepackage{selnolig}  % disable illegal ligatures
\fi

\title{On the interplay of motivational characteristics and school grades: The role of Need for Cognition}
\author{Anja Strobel\textsuperscript{1*}, Alexander Strobel\textsuperscript{2*}, Franzis Preckel\textsuperscript{3}, \& Ricarda Steinmayer\textsuperscript{4}}
\date{}


\shorttitle{NFC, Ability Self-Concept and School Grades}

\authornote{

{[}adjust according to APA style{]}
* These authors contributed equally to this work.
{[}add acknowledgements and funding if applicable{]}

Correspondence concerning this article should be addressed to Anja Strobel, Department of Psychology, Chemnitz University of Technology, 09120 Chemnitz, Germany. E-mail: \href{mailto:anja.strobel@psychologie.tu-chemnitz.de}{\nolinkurl{anja.strobel@psychologie.tu-chemnitz.de}}

}

\affiliation{\vspace{0.5cm}\textsuperscript{1} Department of Psychology, Chemnitz University of Technology, Chemnitz, Germany\\\textsuperscript{2} Faculty of Psychology, Technische Universität Dresden, Dresden, Germany\\\textsuperscript{3} Department of Psychology, University of Trier, Trier, Germany\\\textsuperscript{4} Department of Psychology, TU Dortmund University, Dortmund, Germany}

\abstract{
\ldots{}
}



\begin{document}
\maketitle

In recent decades, a great deal of research has been conducted on the prediction of school performance. Meta-analyses indicate that intelligence is the strongest predictor for academic achievement (e.g., Deary, Strand, Smith, \& Fernandes, 2007; Kriegbaum, Becker, \& Spinath, 2018). Still, motivational variables have consistently been found to also have predictive value for school performance (e.g., Kriegbaum et al., 2018; Steinmayr, Weidinger, Schwinger, \& Spinath, 2019) and concepts like ability self-concept, hope for success and fear of failure, interest and values are well known and equally established indicators (Wigfield \& Cambria, 2010; e.g., Wigfield \& Eccles, 2000) that are subsumed under the umbrella term of achievement motivation (Steinmayr et al., 2019).

Over the last years, an additional predictor of academic performance came into the focus of research in this field: Need for Cognition (NFC), the stable intrinsic motivation of an individual to engage in and enjoy challenging intellectual activity (Cacioppo, Petty, Feinstein, \& Jarvis, 1996). According to Investment Theory (Ackerman \& Heggestad, 1997), traits such as NFC determine how individuals invest their cognitive resources and how they deal with cognitively challenging material. It has been shown that NFC is related to academic performance in different stages of academic life (e.g., Ginet \& Py, 2000; Grass, Strobel, \& Strobel, 2017; Luong et al., 2017; Preckel, 2014; for a meta-analytical review see von Stumm \& Ackerman, 2013) and to behaviors associated with success in learning. As examples, NFC was found to be related to ability self-concept (e.g., Dickhäuser \& Reinhard, 2010; Luong et al., 2017), interest in school (e.g., Preckel, 2014) or deeper processing while learning (Evans, Kirby, \& Fabrigar, 2003; Luong et al., 2017).

The enjoyment of accomplishing something, the interest in task engagement and the intrinsic value of working on a task have been suggested to be relevant to learning and academic achievement and have been integrated into models of achievement motivation (e.g., Wigfield \& Eccles, 2000; see also Wigfield \& Cambria, 2010 for a review). Surprisingly, the concept of a more general joy of thinking, that is NFC, has not yet been investigated systematically together with established motivational indicators, especially in longitudinal studies, or integrated into models for the prediction of performance in school.

Only last year, a large longitudinal study examined intelligence, the Big Five, a range of different motivational measures together with NFC in order to determine their value in predicting school performance (Lavrijsen, Vansteenkiste, Boncquet, \& Verschueren, 2021). Their results showed intelligence and NFC to be the strongest predictors of school performance. The ability self-concept was the best predictor within the group of motivational variables. This underscores the importance to consider NFC along with established predictors in gaining a comprehensive picture of the prediction of school grades.

To follow-up on these findings and to provide new insights in the interplay of school performance, NFC and motivational variables, we examined the incremental value of NFC, considering well-established motivational constructs as well as prior achievement in the prediction of school grades across different subjects in a longitudinal approach in a sample of secondary school children.

\hypertarget{achievement-motivation-and-its-relation-to-school-performance}{%
\subsection{Achievement Motivation and its relation to school performance}\label{achievement-motivation-and-its-relation-to-school-performance}}

Achievement motivation is operationalized through various variables and can be seen as an essential predictor of academic achievement (e.g., Hattie, 2009; Steinmayr \& Spinath, 2009; Wigfield \& Cambria, 2010). Well-established concepts such as ability self-concept, hope for success and fear of failure, or variables such as interests and values can be found under this term (Steinmayr et al., 2019). They have found their way into essential models of achievement motivation (Kriegbaum et al., 2018; e.g., Wigfield \& Eccles, 2000), which is why they were included in this study as important motivational indicators. They are briefly introduced below.

\emph{Ability Self-concept.} Ability self-concept can be described as generalized or subject-specific ability perceptions that students acquire on the basis of competence experiences in the course of their academic life (Möller \& Köller, 2004). They thus reflect cognitive representations of one's level of ability (H. W. Marsh, 1990). Such ability perceptions of students affect their academic performance (e.g., Wigfield \& Eccles, 2000). A meta-analysis found moderate correlations with academic achievement {[}\(r=.34\); Huang (2011){]}, whereas the association was lower (\(r~.20\)) when controlled for prior achievement (e.g., Herbert W. Marsh \& Martin, 2011). Steinmayr et al. (2019) demonstrated that among several motivational indicators, domain-specific ability self-concept was the strongest predictors of school performance. Moreover, ability self-concepts and school performance influence each other and can thus mutually reinforce or weaken each other (e.g., Guay, Marsh, \& Boivin, 2003).

\emph{Hope for Success/Fear of Failure.} Murray (1938) considered the Need for Achievement as one of the basic human needs and as a relatively stable personality trait. His concept was extended by McClelland, Atkinson, Clark, and Lowell (1953), who differentiated the achievement motives hope for success (the belief of being able to succeed accompanied by the experience of positive emotions) and fear of failure (worry about failing in achievement situations and the experience of negative emotions). Such affective tendencies in the context of achievement motivation are reflected, for instance, in the choice of task difficulty, affinity for risk, and quality of task completion (Diseth \& Martinsen, 2003). Hope for success may facilitate knowledge acquisition, whereas fear of failure may impede it (Diseth \& Martinsen, 2003). A meta-analysis found achievement motivation in the sense of hope for success weakly to moderately positively related to academic achievement {[}\(r=.26\); Robbins et al. (2004){]}. For the association of fear of failure and academic achievement, findings from individual studies suggest a relationship of similar magnitude but in a different direction {[}e.g., \(r=-.26\); Dickhäuser, Dinger, Janke, Spinath, and Steinmayr (2016){]}.

\emph{Task values - Interest.} Another important motivational indicator that was also included in the influential model of Wigfield and Eccles (2000), describes task values. Such task values focus on importance, perceived utility, and interest in a task (cf. Jacobs, Lanza, Osgood, Eccles, \& Wigfield, 2002). Specifically on the domain of interest, a number of papers are available on the relationship with school performance, with correlations being in a low to moderate range (for an overview, see Steinmayr et al., 2019). A meta-analysis on the relationship between interest and achievement found moderate positive correlations between these two variables (Schiefele, Krapp, \& Winteler, 1992).

\hypertarget{need-for-cognition-and-academic-performance}{%
\subsection{Need for Cognition and academic performance}\label{need-for-cognition-and-academic-performance}}

NFC describes the stable intrinsic motivation of an individual to engage in and enjoy challenging intellectual activity (Cacioppo et al., 1996). While individuals with lower NFC scores tend to rely more on other people, cognitive heuristics or social comparisons in decision making, individuals with higher NFC scores show a tendency to seek, acquire and reflect on information (Cacioppo et al., 1996). NFC, mirroring the typical cognitive performance of a person, has been shown to be rather modestly related to intelligence and its fluid (Fleischhauer et al., 2010) and crystallized (Stumm \& Ackerman, 2013) components.

NFC correlates with academic performance NFC across different stages of school and university: For example, Preckel (2014) reported a weak positive correlation primarily for math in secondary school. Ginet and Py (2000) found a mean correlation of \(r=.33\) between NFC and school performance across all school years studied, with lower correlations in earlier and higher correlations in later school years, a pattern that can also be found in Luong et al. (2017). Colling, Wollschläger, Keller, Preckel, and Fischbach (2022) also report differences in the strength of the correlations with school performance, here depending on the type of school, with the associations between NFC and performance being strongest in the highest and weakest in the lowest school track. As regards university, low to medium correlations were found for NFC and average grades (see Richardson, Abraham, \& Bond, 2012; Stumm \& Ackerman, 2013). A similar picture emerges for the correlation of NFC and university entrance tests (Cacioppo \& Petty, 1982; Olson, Camp, \& Fuller, 1984; Tolentino, Curry, \& Leak, 1990).

Concerning the interplay of intelligence and NFC in the context of school performance, Strobel, Behnke, Grass, and Strobel (2019) found that reasoning ability and NFC both significantly predicted higher grade point average (GPA). Interestingly, NFC also moderated the relation between intelligence and GPA: at higher levels of NFC, the relation of reasoning ability and GPA was diminished. Although this finding requires independent replication, it could point to a potentially compensating effect of NFC.

\hypertarget{nfc-and-motivational-aspects-of-learning}{%
\subsection{NFC and motivational aspects of learning}\label{nfc-and-motivational-aspects-of-learning}}

The increased willingness to invest mental effort and attention in task and information processing that is typical for individuals with higher NFC is also associated with positive correlations to various traits, behaviours and indicators relevant to learning. Evans et al. (2003) found associations of NFC with deeper processing while learning. Dickhäuser and Reinhard (2010) reported strong associations of NFC with the general ability self-concept and smaller correlations with subject-specific ability self-concepts. Luong et al. (2017) not only reported moderate to high correlations of NFC with aspects of the ability self-concept, but also with learning orientation, processing depth and the desire to learn from mistakes. Preckel (2014) found medium correlations of NFC with learning goals and interest in various school subjects (for the latter association, see also Keller et al., 2019). Furthermore, Elias and Loomis (2002) found NFC and efficacy beliefs to be moderately correlated. Their results suggested that the relationship between NFC and GPA was mediated by efficacy beliefs, in a way that individuals with higher NFC had higher efficacy belief which in turn had a positive effect on academic performance. (Diseth \& Martinsen, 2003) examined another indicator of performance motivation: In a student sample, they found a high positive correlation between NFC and hope for success and a medium negative relationship between NFC and fear of failure. Comparable findings are also reported by Bless, Wänke, Bohner, Fellhauer, and Schwarz (1994). In a large sample of 7th grade students, Lavrijsen et al. (2021) found a strong correlation with performance motivation and no relation of NFC to fear of failure.

Several studies examined NFC along with other motivational variables and found NFC to explain variance in academic performance beyond established motivational variables such as learning orientation or academic self-concept (Keller et al., 2019; Luong et al., 2017). Meier, Vogl, and Preckel (2014) examined potential predictors of the attendance of a gifted class. They found that NFC, compared to other motivational constructs like academic interests and goal orientations, significantly predicted the attendance of a gifted class even when controlling for cognitive ability and other factors like parental education level or academic self-concept. Lavrijsen et al. (2021) examined the predictive value of intelligence, personality (Big Five and NFC) and different motivational constructs for school performance and found intelligence, NFC and the ability self-concept to be the most strongest predictors of math grades and performance in standardized math tests.

\hypertarget{the-present-study}{%
\subsection{The present study}\label{the-present-study}}

All in all, NFC has been proven to be a very promising predictor of school performance over and above other motivational constructs. Yet, so far the evidence on its incremental predictive value is limited by the mainly cross-sectional nature of available studies and by the fact that only a few school subjects were considered. Furthermore, up to now, prior achievement was not integrated as performance predictor in studies examining NFC. This is a limitation insofar as besides students' cognitive abilities their prior achievement could be shown to be a relevant predictor of academic performance (e.g., Hailikari, Nevgi, \& Komulainen, 2007; Steinmayr et al., 2019).

With the present study, we aim at adding to the existing body of research by examining NFC, motivational indicators (ability self-concept, hope for success and fear of failure, interests, each of them general and subject-specific) and school grades (GPA, German, math, physics, and chemistry) at two points of time. By applying latent change score modelling, we will be able to determine the influence of our different predictors on the change of school performance over time. At the same time, mutual influences of changes in school performance, NFC and motivational constructs can be detected (i.e., correlated change). We examine the following hypotheses and research questions:

\begin{enumerate}
\def\labelenumi{\arabic{enumi}.}
\tightlist
\item
  What is the incremental value of Need for Cognition in the prediction of school performance over and above different motivational constructs and prior achievement in school?
\item
  Is Need for Cognition able to predict changes in school achievement over time?
\item
  Are changes in motivational variables, Need for Cognition and school performance related over time?
\end{enumerate}

\hypertarget{methods}{%
\section{Methods}\label{methods}}

\hypertarget{openness-and-transparency}{%
\subsection{Openness and transparency}\label{openness-and-transparency}}

We report how we determined our sample size, all data exclusions, all manipulations, and all measures in the study (cf. Simmons, Nelson, \& Simonsohn, 2012) and follow JARS (Publications and Journal Article Reporting Standards (n.d.)).
Data were analyzed using R, version 4.1.1 (for details see below, \emph{Statistical analyses}).
All data and code for reproducing our analyses are permanently and openly accessible at \url{https://github.com/alex-strobel/NFC-Grades}.
This study was not preregistered.

\hypertarget{participants}{%
\subsection{Participants}\label{participants}}

Sample size was determined by pragmatic considerations, i.e., to collect as many participants given existing time constraints and the longitudinal nature of the project.
We eventually managed to recruit a sample of \(N\) = 277 participants (60\% women) at the first measurement occasion (T1) of which \(N\) = 251 participants (61\% women) also took part at the second measurement occasion (T2) that took place 53-59 weeks later.
Age range was 14-19 years (median = 17 years) at T1 and 15-20 years (median = 18 years) at T2.
With the sample size accomplished at T2, we were able to detect correlations of \emph{r} \(\ge\) .18 at \(\alpha\) = .05 (two-sided) and 1-\(\beta\) = .80.
Yet, we tried to impute missing values to raise power (see below, \emph{Statistical analyses}).

\hypertarget{material}{%
\subsection{Material}\label{material}}

We used the following self-report measures to assess the measures of interest for the present study.

\emph{School Grades} in general, i.e., Grade Point Average (GPA), and grades in German, math, chemistry, and physics were assessed via self-report. In Germany, school grades range from 1 (excellent) to 6 (insufficient). For better interpretability, we reversed this coding via \(6 - grade\), so the values we used for statistical analyses ranged from 0 (insufficient) to 5 (excellent).

\emph{Need for Cognition} (NFC) was assessed with the 16-item short version of the German NFC scale (Bless et al., 1994). Responses to each item (e.g., ``Thinking is not my idea of fun'', recoded) were recorded on a four-point scale ranging from -3 (completely disagree) to +3 (completely agree) and were summed to the total NFC score. The scale has a comparably high internal consistency, Cronbach's \(\alpha\) \textgreater{} .80 (Bless et al., 1994; Fleischhauer et al., 2010), and retest reliability, \(r_{tt} = .83\) across 8 to 18 weeks (Fleischhauer, Strobel, \& Strobel, 2015).

\emph{Hope for Successs} and \emph{Fear of Failure} were assessed using the Achievement Motive Scales (Gjesme \& Nygard, 2006; German version: Göttert \& Kuhl, 1980). For the present study, we used a short form measuring each construct with seven items. All items were answered on a four-point scale ranging from 1 (does not apply at all) to 4 (fully applies). Example items for the two scales are ``Difficult problems appeal to me'' and ``Matters that are slightly difficult disconcert me''. Both scales exhibit high internal consistencies, Cronbach's \(\alpha\ge.85\) (Steinmayr \& Spinath, 2009).

The \emph{Ability Self-Concept} in school in general and in the four subjects German, math, physics, and chemistry were assessed with four items per domain using the Scales for the Assessment of Academic Self-Concept (Schöne, Dickhäuser, Spinath, \& Stiensmeier-Pelster, 2002) (example item: ``I can do well in \ldots{} (school, math, German, physics, chemistry).''). Items were answered on a 5-point scale ranging from 1 () to 5 (). The scales' internal consistency, Cronbach's \(\alpha\ge.80\), and retest reliability, \(r_{tt}\ge.59\) across six months, can be considered as high.

\emph{Interest} in school in general and in the above four subjects were measured using Interest subscales of the Scales for the Assessment of Subjective Values in School (Steinmayr \& Spinath, 2010). Answers to three items per domain (example item: ``How much do you like \ldots{} (school, math, German, physics, chemistry).'') were recorded on a 5-point scale ranging from 1 (\ldots{}) to 5 (\ldots{}). The scales have high internal consistency, Cronbach's \(\alpha\ge.89\), and retest reliability, \(r_{tt} = .72\) across six months (Steinmayr \& Spinath, 2010).

\hypertarget{procedure}{%
\subsection{Procedure}\label{procedure}}

\ldots{}

\hypertarget{statistical-analysis}{%
\subsection{Statistical analysis}\label{statistical-analysis}}

We used \emph{RStudio} (Version 2021.9.0.351, RStudio Team, 2016) with R (Version 4.1.1; R Core Team, 2018) and the R-packages \emph{lavaan} (Version 0.6.10; Rosseel, 2012), \emph{naniar} (Version 0.6.1; Tierney, Cook, McBain, \& Fay, 2021), \emph{psych} (Version 2.1.9; Revelle, 2018), and \emph{pwr} (Version 1.3.0; Champely, 2018). This manuscript was created using RMarkdown with the packages \emph{papaja} (Version 0.1.0.9997, Aust \& Barth, 2018), \emph{knitr} (Version 1.37, Xie, 2015), and \emph{shape} (Version 1.4.6, Soetaert, 2018). Additionally, the packages \emph{renv} (Version 0.14.0, Ushey, 2021) and \emph{here} (Version 1.0.1, Müller, 2020) were employed to enhance the reproducibility of the present project (see \url{https://github.com/alex-strobel/NFC-Grades}).

First the variables were separated into four sets, each containing the T1 and T2 measurements of the variables Hope for Success (HfS), Fear of Failure (FoF), and Need for Cognition (NFC) as well as either GPA, overall ability self-concept regarding school, and general interest in school, or domain-specific grades, ability self-concept and interest in German, math, physics, and chemistry. All measures were initially analyzed with regard to descriptive statistics, reliability (retest-reliability \(r_{tt}\) as well as Cronbach's \(\alpha\)), and possible deviation from univariate and multivariate normality. Almost all relevant variables deviated from univariate normality as determined using Shapiro-Wilks tests with a threshold of \(\alpha\) = .20, all \(p\le\) .089 except for NFC at T2, \emph{p} = .461. Also, there was deviation from multivariate normality as determined using Mardia tests, all \(p_{skew}\) and \(p_{kurtosis}\) \textless{} .001. Therefore, we used more robust variants for the statistical tests to be performed, i.e., Spearman rank correlations (\(r_s\)) for correlation analyses and Robust Maximum Likelihood (MLR) for regression analyses and latent change score modeling.

Possible differences between the measurement occasions T1 and T2 were descriptively assessed via boxplots, with overlapping notches---that can roughly be interpreted as 95\% confidence intervals of a given median---pointing to noteworthy differences. Otherwise differences between time points were not considered further given the scope of the present report. Correlation analyses were performed separately for the five sets of data (see Table 1 and Supplementary Tables S1 to S4). Where appropriate, evaluation of statistical significance was based on 95\% confidence intervals (CI) that did not include zero. Evaluation of effect sizes of correlations was based on the empirically derived guidelines for personality and social psychology research provided by Gignac and Szodorai (2016), i.e., correlations were regarded as small for \(r < .20\), as medium for .20 \(\le r \le\) .30, and as large for \(r > .30\).

To examine which variables measured at T1 would be significant predictors of school grades at T2, we ran a five regression analyses with the GPA and the four subject-specific grades as criterion and used the results of the first regression analysis (with the domain-general Ability Self-Concept, Interest in School, Hope for Success and Fear of failure, and NFC measured at T1 as predictors and GPA at T2 as criterion) to select the variables for latent change score modeling. Significant predictors in this model were used for all latent change score models even if for certain subjects, the predictors were not significant in the respective regression models. Regression models were fitted via \emph{lavaan}, using MLR as estimation technique and---because missing data were missing completely at random (MCAR), all \(p\ge\) .169---the Full-Information Maximum Likelihood (FIML) approach to impute missing values. Due to missing patterns, this resulted in an effective sample size of \(N\) = 271-276. To assess whether a model that included NFC was superior to a model that included established predictors of academic achievement, we (1) evaluated the fit of the respective models based on the recommendations by Hu and Bentler (1999), with values of CFI \(\ge\) .95, RMSEA \(\le\) .06, and SRMR \(\le\) 0.08 indicating good model fit, and (2) performed \(\chi^2\)-difference tests between the former and the latter model (and all other variables' loadings fixed to zero).

In the final step, latent change score modeling was applied. In this approach (see Kievit et al., 2018), one can examine (1) whether true change in a variable has occurred via a latent change score that is modeled from the respective measurements of this variable at different measurement occasions, here T1 and T2, (2) to what extent the change in a variable is a function of the measurement of the \emph{same} variable at T1 (self-feedback) and (3) to what extent the change in this variable is a function of the measurement of \emph{other} variables in the model at T1 (cross-domain coupling). Thereby, cross-domain effects, i.e., whether the change in one domain (e.g., school grades) is a function of the baseline score of another (e.g., NFC) and vice versa could be examined. In addition, correlated change in the variables of interest can be examined, i.e., to what extent does the change in one variable correlate with the change in another variable. Fig. 1A provides an example of a bivariate latent change score model. For latent change score modeling, again MLR estimation and imputation of missing values via FIML was employed.

\hypertarget{results}{%
\section{Results}\label{results}}

\hypertarget{domain-general-grades}{%
\subsection{Domain-general grades}\label{domain-general-grades}}

\begin{table}[tbp]

\begin{center}
\begin{threeparttable}

\caption{\label{tab:corr}Spearman correlations and descriptive statistics of the variables in the analyses on overall school grades}

\footnotesize{

\begin{tabular}{lcccccccccccc}
\toprule
 & \multicolumn{1}{c}{GRD1} & \multicolumn{1}{c}{ASC1} & \multicolumn{1}{c}{INT1} & \multicolumn{1}{c}{HFS1} & \multicolumn{1}{c}{FOF1} & \multicolumn{1}{c}{NFC1} & \multicolumn{1}{c}{GRD2} & \multicolumn{1}{c}{ASC2} & \multicolumn{1}{c}{INT2} & \multicolumn{1}{c}{HFS2} & \multicolumn{1}{c}{FOF2} & \multicolumn{1}{c}{NFC2}\\
\midrule
GRD1 & \textit{—} & .58 & .38 & .34 & -.24 & .44 & \textbf{\textit{.75}} & .52 & .34 & .40 & -.23 & .49\\
ASC1 &  & \textit{.83} & .49 & .37 & -.27 & .38 & .50 & \textbf{\textit{.60}} & .32 & .34 & -.18 & .26\\
INT1 &  &  & \textit{.88} & .32 & -.09 & .35 & .44 & .47 & \textbf{\textit{.65}} & .31 & -.05 & .26\\
HFS1 &  &  &  & \textit{.86} & -.30 & .62 & .32 & .38 & .26 & \textbf{\textit{.57}} & -.17 & .50\\
FOF1 &  &  &  &  & \textit{.88} & -.42 & -.17 & -.28 & -.14 & -.29 & \textbf{\textit{.59}} & -.43\\
NFC1 &  &  &  &  &  & \textit{.89} & .46 & .43 & .25 & .62 & -.32 & \textbf{\textit{.71}}\\
GRD2 &  &  &  &  &  &  & \textit{—} & .53 & .34 & .41 & -.18 & .48\\
ASC2 &  &  &  &  &  &  &  & \textit{.84} & .53 & .45 & -.25 & .46\\
INT2 &  &  &  &  &  &  &  &  & \textit{.88} & .31 & -.05 & .34\\
HFS2 &  &  &  &  &  &  &  &  &  & \textit{.87} & -.28 & .66\\
FOF2 &  &  &  &  &  &  &  &  &  &  & \textit{.90} & -.39\\
NFC2 &  &  &  &  &  &  &  &  &  &  &  & \textit{.89}\\ \midrule
Mean & 3.30 & 3.55 & 3.25 & 2.92 & 1.86 & 4.46 & 3.46 & 3.62 & 3.41 & 2.72 & 1.71 & 4.69\\
SD & 0.55 & 0.54 & 0.83 & 0.57 & 0.61 & 0.84 & 0.52 & 0.56 & 0.82 & 0.56 & 0.61 & 0.87\\
Min & 2.00 & 1.75 & 1.00 & 1.14 & 1.00 & 2.19 & 2.10 & 2.25 & 1.00 & 1.00 & 1.00 & 2.50\\
Max & 5.00 & 5.00 & 5.00 & 4.00 & 4.00 & 6.94 & 5.00 & 5.00 & 5.00 & 4.00 & 3.71 & 6.88\\
Skew & 0.17 & 0.09 & -0.27 & -0.23 & 0.45 & 0.16 & 0.31 & 0.33 & -0.21 & -0.02 & 0.89 & 0.07\\
Kurtosis & -0.09 & 0.24 & -0.37 & -0.07 & -0.34 & 0.14 & -0.11 & -0.14 & -0.42 & 0.17 & 0.47 & -0.45\\
\bottomrule
\addlinespace
\end{tabular}

}

\begin{tablenotes}[para]
\normalsize{\textit{Note.} \textit{N} = 193-259 due to missings; $p < .05$ for $|r_{s}|$ > .18; coefficients in the diagonal are Cronbach’s $\alpha$, bold-faced coefficients give the 53-59 week retest reliability; GRD = Grade Point Average, ASC = Overall Ability Self-Concept, INT = Overall Interest in School, HFS = Hope for Success, FOF = Fear of Failure, NFC = Need for Cognition at measurement occasion 1, and 2, respectively}
\end{tablenotes}

\end{threeparttable}
\end{center}

\end{table}

Table \ref{tab:corr} gives the descriptive statistics and intercorrelations of the variables of interest in this analysis step, i.e., the T1 and T2 measurements of GPA, domain-general ability self-concept, and general interest in school as well as the variables Hope for Success, Fear of Failure, and NFC. As can be seen in the diagonal and the upper right of the correlation table, all variables exhibited good internal consistency, Cronbach's \(\alpha\ge\) .83, and retest reliability, \(r_{tt}\ge\) .56. Among the predictors at T1, GPA at T1 showed the strongest relation to GPA at T2, \(r_{s}=.75\), followed by the domain-general ability self-concept, \(r_{s}=.53\), and NFC at T1, \(r_{s}=.46\), all \(p<\) .001. The other variables at T1 showed significant correlations with GPA at T2 as well, \(|r_{s}|\ge.20\), \(p\le.004\).

\begin{table}[tbp]

\begin{center}
\begin{threeparttable}

\caption{\label{tab:mr}Results of the multiple regression of school grades measured at T2 on predictors measured at T1}

\begin{tabular}{lrrrrrr}
\toprule
 & $B$ & $SE$ & $CI.LB$ & $CI.UB$ & $\beta$ & $p$\\
\midrule
Intercept & 0.488 & 0.231 & 0.034 & 0.941 & .906 & .035\\
GPA & 0.606 & 0.061 & 0.485 & 0.726 & .616 & < .001\\
Ability Self-Concept & 0.116 & 0.054 & 0.010 & 0.222 & .117 & .031\\
Interest & 0.057 & 0.031 & -0.005 & 0.118 & .087 & .072\\
Hope for Success & -0.028 & 0.050 & -0.126 & 0.070 & -.029 & .578\\
Fear of Failure & 0.013 & 0.039 & -0.063 & 0.089 & .015 & .733\\
Need for Cognition & 0.089 & 0.040 & 0.012 & 0.167 & .140 & .024\\
\bottomrule
\addlinespace
\end{tabular}

\begin{tablenotes}[para]
\normalsize{\textit{Note.} $N$ = 276; coefficients are unstandardized slopes $B$ with their standard errors $SE$ and 95\% confidence intervals ($CI.LB$ = lower bound, $CI.UB$ = upper bound), $\beta$ is the standardized slope and $p$ the respective $p$-vcalues}
\end{tablenotes}

\end{threeparttable}
\end{center}

\end{table}

A multiple regression analysis involving all measures at T1 (see Table \ref{tab:mr}) showed that apart from GPA at T1, \(B=\) 0.61, 95\% CI {[}0.49, 0.73{]}, \(p< .001\), the only significant predictors were the domain-general ability self-concept, \(B=\) 0.12, 95\% CI {[}0.01, 0.22{]}, \(p=.031\), and NFC, \(B=\) 0.09, 95\% CI {[}0.01, 0.17{]}, \(p=.024\). Model fit was better for a model that included GPA, the ability self-concept, and NFC at T1 (while all other predictors were set to zero), \(\chi^2(3)\) = 3.68, \(p\) .299, CFI = 1.00, RMSEA = .03 with 90\% CI {[}0.00, 0.11{]}, SRMR = .01, than a model that included GPA and the ability self-concept only, \(\chi^2(4)\) = 10.91, \(p\) .028, CFI = 0.96, RMSEA = .08 with 90\% CI {[}0.02, 0.14{]}, SRMR = .02, and a \(\chi^2\)-difference test supported the superiority of the former compared to the latter model, \(\chi^2\)(1) = 6.34, \(p\) = .012.

We therefore further examined a trivariate latent change score model involving school grades, the ability self-concept, and NFC. Fig. 1B gives the results of the latent change score modeling with regard to the prediction of change and correlated change in overall school grades, i.e., GPA. While the best predictor of change on GPA was GPA at T1 (i.e., self-feedback), \(B=\) -0.37, 95\% CI {[}-0.48, -0.25{]}, \(p< .001\), \(\beta=\) -.55, there was also evidence for cross-domain coupling, as the overall ability self-concept and NFC at T1 also significantly predicted change in GPA, \(B=\) 0.13, 95\% CI {[}0.02, 0.24{]}, \(p=.020\), \(\beta=\) .19, and \(B=\) 0.08, 95\% CI {[}0.02, 0.15{]}, \(p=.009\), \(\beta=\) .19, respectively. Correlated change was observed for GPA and the ability self-concept, \(B=\) 0.03, 95\% CI {[}0.01, 0.05{]}, \(p=.001\), \(\beta=\) .22, and the ability self-concept and NFC, \(B=\) 0.05, 95\% CI {[}0.02, 0.08{]}, \(p.001\), \(\beta=\) .22, while the correlated changes in GPA and NFC did not reach significance, \(B=\) 0.03, 95\% CI {[}0.00, 0.05{]}, \(p=.053\), \(\beta=\) .14.

\begin{figure}
\centering
\includegraphics{"Fig1.jpg"}
\caption{Latent change score models. (A) Example of a bivariate latent change score model (for details see text); legend to lines: dotted = loadings fixed to zero, red = self-feedback \(\beta\), blue = cross-domain coupling \(\gamma\), grey = correlation \(\phi\) of predictors at T1, green = correlated change \(\rho\); (B) Grade Point Average (GPA) and (C) to (F) subject-specific changes in grades at T2 (indicated by prefix \(\Delta\)) as predicted by their respective T1 levels as well as by Need for Cognition (NFC) and (overall as well as subject specific) Ability Self-Concept (ASC) at T1; coefficients are standardized coefficients.}
\end{figure}

\hypertarget{domain-specific-grades}{%
\subsection{Domain-specific grades}\label{domain-specific-grades}}

For the four subjects examined, i.e., German, math, physics, and chemistry, similar results were obtained with regard to correlation analyses (see Supplementary Tables Sx to Sy). As regards multiple regression analyses (see Supplementary Table Sz), for all subjects, grades at T2 were significant predictors of grades at T2, \(p<.001\). The subject-specific ability self concept at T1 was a significant predictor of grades at T2 in German only, \(B=\) 0.29, 95\% CI {[}0.15, 0.43{]}, \(p< .001\). NFC at T1 was a significant predictor of T2 grades in German, \(B=\) 0.18, 95\% CI {[}0.05, 0.32{]}, \(p=.007\) and physics, \(B=\) 0.22, 95\% CI {[}0.07, 0.37{]}, \(p=.004\). In both cases, models with NFC as predictor together with grades at T1 and ability self-concept were superior to models with grades at T1 and ability self-concept only, German: \(\chi^2\)(1) = 9.31, \(p\) = .002, physics: \(\chi^2\)(1) = 13.49, \(p\) = \textless{} .001.

As regards the latent change score models, there was evidence for significant self-feedback for all subjects, all \(p<.001\). With regard to the subject-specific ability self-concept, cross-domain coupling with changes in grades was observed for German, \(B=\) 0.28, 95\% CI {[}0.16, 0.40{]}, \(p< .001\), \(\beta=\) .36, and chemistry, \(B=\) 0.09, 95\% CI {[}0.00, 0.18{]}, \(p=.042\), \(\beta=\) .14. NFC at T1 showed cross-domain coupling with grades at T2 for German, \(B=\) 0.13, 95\% CI {[}0.04, 0.21{]}, \(p=.005\), \(\beta=\) .17, physics, \(B=\) 0.23, 95\% CI {[}0.13, 0.33{]}, \(p< .001\), \(\beta=\) .24, and chemistry, \(B=\) 0.10, 95\% CI {[}0.00, 0.20{]}, \(p=.047\), \(\beta=\) .13. Correlated change between grades and the subject-specific ability self-concept was observed for all subjects, while correlated change between grades and NFC was observed for German, math, and physics only (see Fig. 1C-F).

\hypertarget{discussion}{%
\section{Discussion}\label{discussion}}

The present study was conducted in order to \ldots{} It would be most convenient if you could write the Discussion directly in the R Markdown document. If you want to highlight something, set it in \emph{italics} via asterisks before and after the word/phrase to be highlighted. And please enter references as @AuthorYear if you want them to appear directly in the text or as {[}@AuthorYear{]} to give them in parentheses. Example: Diseth and Martinsen (2003) found \ldots{} (see also Robbins et al., 2004). If you cite new references that not already appeared in the Introduction, just paste them at the end of the Discussion as plain text like:

Weber, H. S., Lu, L., Shi, J., and Spinath, F. M. (2013). The roles of cognitive and motivational predictors in explaining school achievement in elementary school. Learn. Individ. Differ. 25, 85--92. doi: 10.1016/j.lindif.2013.03.008

Format does not matter.

\hypertarget{subheading-1}{%
\subsection{Subheading 1}\label{subheading-1}}

Our result show that \ldots{}

\hypertarget{subheading-2}{%
\subsection{Subheading 2}\label{subheading-2}}

\ldots{}

\hypertarget{conclusion}{%
\subsection{Conclusion}\label{conclusion}}

Taken together, the present study provides evidence that \ldots{}

\newpage

\hypertarget{references}{%
\section{References}\label{references}}

\begingroup
\setlength{\parindent}{-0.5in}
\setlength{\leftskip}{0.5in}

\hypertarget{refs}{}
\begin{CSLReferences}{1}{0}
\leavevmode\vadjust pre{\hypertarget{ref-Ackerman1997}{}}%
Ackerman, P. L., \& Heggestad, E. D. (1997). Intelligence, personality, and interests: Evidence for overlapping traits. \emph{Psychological Bulletin}, \emph{121}(2), 219.

\leavevmode\vadjust pre{\hypertarget{ref-R-papaja}{}}%
Aust, F., \& Barth, M. (2018). \emph{{papaja}: {Create} {APA} manuscripts with {R Markdown}}. Retrieved from \url{https://github.com/crsh/papaja}

\leavevmode\vadjust pre{\hypertarget{ref-Bless1994}{}}%
Bless, H., Wänke, M., Bohner, G., Fellhauer, R. L., \& Schwarz, N. (1994). Need for {C}ognition: {E}ine {S}kala zur {E}rfassung von {E}ngagement und {F}reude bei {D}enkaufgaben {[}{N}eed for {C}ognition: A scale measuring engagement and happiness in cognitive tasks{]}. \emph{Zeitschrift {f}{ü}r Sozialpsychologie}, \emph{25}, 147--154.

\leavevmode\vadjust pre{\hypertarget{ref-Cacioppo1982}{}}%
Cacioppo, J. T., \& Petty, R. E. (1982). The need for cognition. \emph{Journal of Personality and Social Psychology}, \emph{42}, 116--131.

\leavevmode\vadjust pre{\hypertarget{ref-Cacioppo1996}{}}%
Cacioppo, J. T., Petty, R. E., Feinstein, J. A., \& Jarvis, W. B. G. (1996). Dispositional differences in cognitive motivation: The life and times of individuals varying in {N}eed for {C}ognition. \emph{Psychological Bulletin}, \emph{119}(2), 197--253. \url{https://doi.org/10.1037/0033-2909.119.2.197}

\leavevmode\vadjust pre{\hypertarget{ref-R-pwr}{}}%
Champely, S. (2018). \emph{Pwr: Basic functions for power analysis}. Retrieved from \url{https://CRAN.R-project.org/package=pwr}

\leavevmode\vadjust pre{\hypertarget{ref-Colling2021}{}}%
Colling, J., Wollschläger, R., Keller, U., Preckel, F., \& Fischbach, A. (2022). Need for cognition and its relation to academic achievement in different learning environments. \emph{Learning and Individual Differences}, \emph{93}, 102110. https://doi.org/\url{https://doi.org/10.1016/j.lindif.2021.102110}

\leavevmode\vadjust pre{\hypertarget{ref-Deary2007}{}}%
Deary, I. J., Strand, S., Smith, P., \& Fernandes, C. (2007). Intelligence and educational achievement. \emph{Intelligence}, \emph{35}(1), 13--21. \url{https://doi.org/10.1016/j.intell.2006.02.001}

\leavevmode\vadjust pre{\hypertarget{ref-Dickhaeuser2016}{}}%
Dickhäuser, O., Dinger, F. C., Janke, S., Spinath, B., \& Steinmayr, R. (2016). A prospective correlational analysis of achievement goals as mediating constructs linking distal motivational dispositions to intrinsic motivation and academic achievement. \emph{Learning and Individual Differences}, \emph{50}, 30--41. \url{https://doi.org/10.1016/j.lindif.2016.06.020}

\leavevmode\vadjust pre{\hypertarget{ref-Dickhaeuser2010}{}}%
Dickhäuser, O., \& Reinhard, M.-A. (2010). How students build their performance expectancies: The importance of need for cognition. \emph{European Journal of Psychology of Education}, \emph{25}(3), 399--409. \url{https://doi.org/10.1007/s10212-010-0027-4}

\leavevmode\vadjust pre{\hypertarget{ref-Diseth2003}{}}%
Diseth, Å., \& Martinsen, Ø. (2003). Approaches to learning, cognitive style, and motives as predictors of academic achievement. \emph{Educational Psychology}, \emph{23}(2), 195--207. \url{https://doi.org/10.1080/01443410303225}

\leavevmode\vadjust pre{\hypertarget{ref-Elias2002}{}}%
Elias, S. M., \& Loomis, R. J. (2002). Utilizing need for cognition and perceived self-efficacy to predict academic Performance1. \emph{Journal of Applied Social Psychology}, \emph{32}(8), 1687--1702. \url{https://doi.org/10.1111/j.1559-1816.2002.tb02770.x}

\leavevmode\vadjust pre{\hypertarget{ref-Evans2003}{}}%
Evans, C. J., Kirby, J. R., \& Fabrigar, L. R. (2003). Approaches to learning, need for cognition, and strategic flexibility among university students. \emph{British Journal of Educational Psychology}, \emph{73}(4), 507--528.

\leavevmode\vadjust pre{\hypertarget{ref-Fleischhauer2010}{}}%
Fleischhauer, M., Enge, S., Brocke, B., Ullrich, J., Strobel, A., \& Strobel, A. (2010). Same or different? Clarifying the relationship of {N}eed for {C}ognition to personality and intelligence. \emph{Personality \& Social Psychology Bulletin}, \emph{36}(1), 82--96. \url{https://doi.org/10.1177/0146167209351886}

\leavevmode\vadjust pre{\hypertarget{ref-Fleischhauer2015}{}}%
Fleischhauer, M., Strobel, A., \& Strobel, A. (2015). Directly and indirectly assessed {N}eed for {C}ognition differentially predict spontaneous and reflective information processing behavior. \emph{Journal of Individual Differences}, \emph{36}(2), 101--109. \url{https://doi.org/10.1027/1614-0001/a000161}

\leavevmode\vadjust pre{\hypertarget{ref-Gignac2016}{}}%
Gignac, G. E., \& Szodorai, E. T. (2016). Effect size guidelines for individual differences researchers. \emph{Personality and Individual Differences}, \emph{102}, 74--78. \url{https://doi.org/10.1016/j.paid.2016.06.069}

\leavevmode\vadjust pre{\hypertarget{ref-Ginet2000}{}}%
Ginet, A., \& Py, J. (2000). Le besoin de cognition: Une {é}chelle fran{ç}aise pour enfants et ses cons{é}quences au plan sociocognitif. \emph{L'ann{é}e Psychologique}, \emph{100}(4), 585--627.

\leavevmode\vadjust pre{\hypertarget{ref-Gjesme1970}{}}%
Gjesme, T., \& Nygard, R. (2006). \emph{Achievement-related motives: Theoretical considerations and construction of a measuring instrument}. University of Oslo.

\leavevmode\vadjust pre{\hypertarget{ref-Goettert1980}{}}%
Göttert, R., \& Kuhl, J. (1980). AMS --- {A}chievement {M}otives {S}cale von {G}jesme und {N}ygard --- {D}eutsche {F}assung {[}{AMS} --- {G}erman version{]}. In F. Rheinberg \& S. Krug (Eds.), \emph{Motivationsf{ö}rderung im {S}chulalltag {[}{E}nhancement of motivation in school context{]}} (pp. 194--200). G{ö}ttingen: Hogrefe.

\leavevmode\vadjust pre{\hypertarget{ref-Grass2017}{}}%
Grass, J., Strobel, A., \& Strobel, A. (2017). Cognitive investments in academic success: The role of need for cognition at university. \emph{Frontiers in Psychology}, \emph{8}, 790. \url{https://doi.org/10.3389/fpsyg.2017.00790}

\leavevmode\vadjust pre{\hypertarget{ref-Guay2003}{}}%
Guay, F., Marsh, H. W., \& Boivin, M. (2003). Academic self-concept and academic achievement: Relations and causal ordering. \emph{Journal of Educational Psychology}, \emph{95}, 124--136. \url{https://doi.org/10.1037/0022-0663.95.1.124}

\leavevmode\vadjust pre{\hypertarget{ref-Hailikari2007}{}}%
Hailikari, T., Nevgi, A., \& Komulainen, E. (2007). Academic self-beliefs and prior knowledge as predictors of student achievement in mathematics: A structural model. \emph{Educational Psychology}, \emph{28}, 59--71. \url{https://doi.org/10.1080/01443410701413753}

\leavevmode\vadjust pre{\hypertarget{ref-Hattie2009}{}}%
Hattie, J. A. C. (2009). \emph{Visible learning: A synthesis of 800 + meta-analyses on achievement}. Oxford: Routledge.

\leavevmode\vadjust pre{\hypertarget{ref-Hu1999}{}}%
Hu, L. T., \& Bentler, P. M. (1999). Cutoff criteria for fit indexes in covariance structure analysis: Conventional criteria versus new alternatives. \emph{Structural Equation Modeling-A Multidisciplinary Journal}, \emph{6}(1), 11--55. \url{https://doi.org/10.1080/10705519909540118}

\leavevmode\vadjust pre{\hypertarget{ref-Huang2011}{}}%
Huang, C. (2011). Self-concept and academic achievement: A meta-analysis of longitudinal relations. \emph{Journal of School Psychology}, \emph{49}(5), 505--528. \url{https://doi.org/10.1016/j.jsp.2011.07.001}

\leavevmode\vadjust pre{\hypertarget{ref-Jacobs2002}{}}%
Jacobs, J. E., Lanza, S., Osgood, D. W., Eccles, J. S., \& Wigfield, A. (2002). Changes in children's self-competence and values: Gender and domain differences across grades one though twelve. \emph{Child Development}, \emph{73}(2), 509--527. \url{https://doi.org/10.1111/1467-8624.00421}

\leavevmode\vadjust pre{\hypertarget{ref-Keller2016}{}}%
Keller, U., Strobel, A., Wollschläger, R., Greiff, S., Martin, R., Vainikainen, M.-P., \& Preckel, F. (2019). A need for cognition scale for children and adolescents. \emph{European Journal of Psychological Assessment}, \emph{35}(1), 137--149. \url{https://doi.org/10.1027/1015-5759/a000370}

\leavevmode\vadjust pre{\hypertarget{ref-Kievit2018}{}}%
Kievit, R. A., Brandmaier, A. M., Ziegler, G., van Harmelen, A.-L., de Mooij, S. M. M., Moutoussis, M., \ldots{} Dolan, R. J. (2018). Developmental cognitive neuroscience using latent change score models: A tutorial and applications. \emph{Developmental Cognitive Neuroscience}, \emph{33}, 99--117. \url{https://doi.org/10.1016/j.dcn.2017.11.007}

\leavevmode\vadjust pre{\hypertarget{ref-Kriegbaum2018}{}}%
Kriegbaum, K., Becker, N., \& Spinath, B. (2018). The relative importance of intelligence and motivation as predictors of school achievement: A meta-analysis. \emph{Educational Research Review}, \emph{25}, 120--148. \url{https://doi.org/10.1016/j.edurev.2018.10.001}

\leavevmode\vadjust pre{\hypertarget{ref-Lavrijsen2021}{}}%
Lavrijsen, J., Vansteenkiste, M., Boncquet, M., \& Verschueren, K. (2021). Does motivation predict changes in academic achievement beyond intelligence and personality? A multitheoretical perspective. \emph{Journal of Educational Psychology}. \url{https://doi.org/10.1037/edu0000666}

\leavevmode\vadjust pre{\hypertarget{ref-Luong2017}{}}%
Luong, C., Strobel, A., Wollschläger, R., Greiff, S., Vainikainen, M.-P., \& Preckel, F. (2017). Need for cognition in children and adolescents: Behavioral correlates and relations to academic achievement and potential. \emph{Learning and Individual Differences}, \emph{53}, 103--113. \url{https://doi.org/10.1016/j.lindif.2016.10.019}

\leavevmode\vadjust pre{\hypertarget{ref-Marsh1990}{}}%
Marsh, H. W. (1990). Causal ordering of academic self-concept and academic achievement: A multiwave, longitudinal panel analysis. \emph{Journal of Educational Psychology}, \emph{82}, 646--656. \url{https://doi.org/10.1037/0022-0663.82.4.646}

\leavevmode\vadjust pre{\hypertarget{ref-Marsh2011}{}}%
Marsh, Herbert W., \& Martin, A. J. (2011). Academic self-concept and academic achievement: Relations and causal ordering. \emph{British Journal of Educational Psychology}, \emph{81}, 59--77. \url{https://doi.org/10.1348/000709910X50350}

\leavevmode\vadjust pre{\hypertarget{ref-McClelland1953}{}}%
McClelland, D. C., Atkinson, J. W., Clark, R. A., \& Lowell, E. L. \&. (1953). \emph{The achievement motive}. New York: Appleton-Century Crofts.

\leavevmode\vadjust pre{\hypertarget{ref-Meier2014}{}}%
Meier, E., Vogl, K., \& Preckel, F. (2014). Motivational characteristics of students in gifted classes: The pivotal role of need for cognition. \emph{Learning and Individual Differences}, \emph{33}, 39--46. \url{https://doi.org/10.1016/j.lindif.2014.04.006}

\leavevmode\vadjust pre{\hypertarget{ref-Moeller2004}{}}%
Möller, J., \& Köller, O. (2004). Die genese akademischer selbstkonzepte {[}the genesis of academic self-concepts{]}. \emph{Psychologische Rundschau}, \emph{55}(1), 19--27. \url{https://doi.org/10.1026/0033-3042.55.1.19}

\leavevmode\vadjust pre{\hypertarget{ref-R-here}{}}%
Müller, K. (2020). \emph{Here: A simpler way to find your files}. Retrieved from \url{https://CRAN.R-project.org/package=here}

\leavevmode\vadjust pre{\hypertarget{ref-Murray1938}{}}%
Murray, H. A. (1938). \emph{Explorations in personality}. Oxford University Press.

\leavevmode\vadjust pre{\hypertarget{ref-Olson1984}{}}%
Olson, K. R., Camp, C. J., \& Fuller, D. (1984). Curiosity and need for cognition. \emph{Psychological Reports}, \emph{54}(1), 71--74. \url{https://doi.org/10.2466/pr0.1984.54.1.71}

\leavevmode\vadjust pre{\hypertarget{ref-Preckel2014}{}}%
Preckel, F. (2014). Assessing {Need} for {Cognition} in early adolescence: Validation of a german adaption of the {Cacioppo}/{Petty} scale. \emph{European Journal of Psychological Assessment}, \emph{30}(1), 65--72. \url{https://doi.org/10.1027/1015-5759/a000170}

\leavevmode\vadjust pre{\hypertarget{ref-APA2008}{}}%
Publications, A., \& Journal Article Reporting Standards, C. B. W. G. on. (n.d.). Reporting standards for research in psychology: Why do we need them? What might they be? \emph{American Psychologist}, \emph{63}, 839--851. \url{https://doi.org/10.1037/0003-066X.63.9.839}

\leavevmode\vadjust pre{\hypertarget{ref-R-base}{}}%
R Core Team. (2018). \emph{R: A language and environment for statistical computing}. Vienna, Austria: R Foundation for Statistical Computing. Retrieved from \url{https://www.R-project.org/}

\leavevmode\vadjust pre{\hypertarget{ref-R-psych}{}}%
Revelle, W. (2018). \emph{Psych: Procedures for psychological, psychometric, and personality research}. Evanston, Illinois: Northwestern University. Retrieved from \url{https://CRAN.R-project.org/package=psych}

\leavevmode\vadjust pre{\hypertarget{ref-Richardson2012}{}}%
Richardson, M., Abraham, C., \& Bond, R. (2012). Psychological correlates of university students' academic performance: A systematic review and meta-analysis. \emph{Psychological Bulletin}, \emph{138}(2), 353--387. \url{https://doi.org/10.1037/a0026838}

\leavevmode\vadjust pre{\hypertarget{ref-Robbins2004}{}}%
Robbins, S. B., Lauver, K., Le, H., Davis, D., Langley, R., \& Carlstrom, A. (2004). Do psychosocial and study skill factors predict college outcomes? A meta-analysis. \emph{Psychological Bulletin}, \emph{130}, 261--288. \url{https://doi.org/10.1037/0033-2909.130.2.261}

\leavevmode\vadjust pre{\hypertarget{ref-R-lavaan}{}}%
Rosseel, Y. (2012). {lavaan}: An {R} package for structural equation modeling. \emph{Journal of Statistical Software}, \emph{48}(2), 1--36. Retrieved from \url{http://www.jstatsoft.org/v48/i02/}

\leavevmode\vadjust pre{\hypertarget{ref-RStudio}{}}%
RStudio Team. (2016). \emph{RStudio: Integrated development environment for {R}}. Boston, MA: RStudio, Inc. Retrieved from \url{http://www.rstudio.com/}

\leavevmode\vadjust pre{\hypertarget{ref-Schiefele1992}{}}%
Schiefele, U., Krapp, A., \& Winteler, A. (1992). Interest as a predictor of academic achievement: A meta-analysis of research. In K. A. Renninger, S. Hidi, \& A. Krapp (Eds.), \emph{The role of interest in learning and development} (pp. 183--212). Hillsdale, NJ: Lawrence Erlbaum Associates, Inc.

\leavevmode\vadjust pre{\hypertarget{ref-Schoene2002}{}}%
Schöne, C., Dickhäuser, O., Spinath, B., \& Stiensmeier-Pelster, J. (2002). \emph{{Die Skalen zur Erfassung des schulischen Selbstkonzepts (SESSKO) --- Scales for measuring the academic ability self-concept}}. G{ö}ttingen: Hogrefe.

\leavevmode\vadjust pre{\hypertarget{ref-Simmons2012}{}}%
Simmons, J. P., Nelson, L. D., \& Simonsohn, U. (2012). \emph{A 21 word solution}. \url{https://doi.org/10.2139/ssrn.2160588}

\leavevmode\vadjust pre{\hypertarget{ref-R-shape}{}}%
Soetaert, K. (2018). \emph{Shape: Functions for plotting graphical shapes, colors}. Retrieved from \url{https://CRAN.R-project.org/package=shape}

\leavevmode\vadjust pre{\hypertarget{ref-Steinmayr2009}{}}%
Steinmayr, R., \& Spinath, B. (2009). The importance of motivation as a predictor of school achievement. \emph{Learning and Individual Differences}, \emph{19}(1), 80--90. \url{https://doi.org/10.1016/j.lindif.2008.05.004}

\leavevmode\vadjust pre{\hypertarget{ref-Steinmayr2010}{}}%
Steinmayr, R., \& Spinath, B. (2010). {Konstruktion und erste Validierung einer Skala zur Erfassung subjektiver schulischer Werte (SESSW) - {[}Construction and first validation of a scale for the assessment of subjective values in school{]}}. \emph{Diagnostica}, \emph{56}, 195--211. \url{https://doi.org/10.1026/0012-1924/a000023}

\leavevmode\vadjust pre{\hypertarget{ref-Steinmayr2019}{}}%
Steinmayr, R., Weidinger, A. F., Schwinger, M., \& Spinath, B. (2019). The importance of students' motivation for their academic achievement - {R}eplicating and extending previous findings. \emph{Frontiers in Psychology}, \emph{10}. \url{https://doi.org/10.3389/fpsyg.2019.01730}

\leavevmode\vadjust pre{\hypertarget{ref-Strobel2019}{}}%
Strobel, A., Behnke, A., Grass, J., \& Strobel, A. (2019). The interplay of intelligence and need for cognition in predicting school grades: A retrospective study. \emph{Personality and Individual Differences}, \emph{144}, 147--152. \url{https://doi.org/10.1016/j.paid.2019.02.041}

\leavevmode\vadjust pre{\hypertarget{ref-vonStumm2013}{}}%
Stumm, S. von, \& Ackerman, P. (2013). Investment and intellect: A review and meta-analysis. \emph{Psychological Bulletin}, \emph{139}, 841--869. \url{https://doi.org/10.1037/a0030746}

\leavevmode\vadjust pre{\hypertarget{ref-R-naniar}{}}%
Tierney, N., Cook, D., McBain, M., \& Fay, C. (2021). \emph{Naniar: Data structures, summaries, and visualisations for missing data}. Retrieved from \url{https://CRAN.R-project.org/package=naniar}

\leavevmode\vadjust pre{\hypertarget{ref-Tolentino1990}{}}%
Tolentino, E., Curry, L., \& Leak, G. (1990). Further validation of the short form of the need for cognition scale. \emph{Psychological Reports}, \emph{66}(1), 321--322. \url{https://doi.org/10.2466/PR0.66.1.321-322}

\leavevmode\vadjust pre{\hypertarget{ref-R-renv}{}}%
Ushey, K. (2021). \emph{Renv: Project environments}. Retrieved from \url{https://CRAN.R-project.org/package=renv}

\leavevmode\vadjust pre{\hypertarget{ref-Wigfield2010}{}}%
Wigfield, A., \& Cambria, J. (2010). Students' achievement values, goal orientations, and interest: Definitions, development, and relations to achievement outcomes. \emph{Developmental Review}, \emph{30}(1), 1--35. \url{https://doi.org/10.1016/j.dr.2009.12.001}

\leavevmode\vadjust pre{\hypertarget{ref-Wigfield2000}{}}%
Wigfield, A., \& Eccles, J. S. (2000). Expectancy-value theory of achievement motivation. \emph{Contemporary Educational Psychology}, \emph{25}(1), 68--81. \url{https://doi.org/10.1006/ceps.1999.1015}

\leavevmode\vadjust pre{\hypertarget{ref-R-knitr}{}}%
Xie, Y. (2015). \emph{Dynamic documents with {R} and knitr} (2nd ed.). Boca Raton, Florida: Chapman; Hall/CRC. Retrieved from \url{https://yihui.name/knitr/}

\end{CSLReferences}

\endgroup

\newpage


\end{document}
